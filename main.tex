\title{Tarea 0}
\documentclass{article}
\usepackage[utf8]{inputenc}
\usepackage[paperwidth=216mm,paperheight=279mm,left=10mm,top=10mm,bottom=15mm,right=10mm]{geometry}
\usepackage{enumitem} % Más opciones para las listas
\usepackage{hyperref} % Refrencias e hipervínculos
\usepackage{tikz} % Imagenes detras del encabezado
\usepackage{graphicx} % Imagenes 
\usepackage{xcolor} % Colores
    \definecolor{gray}{gray}{0.40}
    \definecolor{mGreen}{RGB}{46,139,87}
    \definecolor{mBlue}{RGB}{23,23,255}
    \definecolor{gray1}{gray}{0.5}
    \definecolor{gray2}{gray}{0.3}
    \definecolor{backgroundColour}{gray}{0.9}
\usepackage{listings}\lstdefinestyle{CStyle}{backgroundcolor=\color{backgroundColour}, commentstyle=\color{gray2}, keywordstyle=\color{mBlue}, numberstyle=\tiny\color{gray1}, stringstyle=\color{orange}, basicstyle=\footnotesize, breakatwhitespace=false, breaklines=true, captionpos=b, keepspaces=true, numbers=left, numbersep=5pt, showspaces=false, showstringspaces=false, showtabs=false,  tabsize=2, language=C}
\usepackage{fancyhdr}\rfoot{\thepage}\cfoot{}\renewcommand{\headrulewidth}{0pt}\pagestyle{fancy} % Estilo de la página

\setlength{\parindent}{0cm}

\renewcommand{\r}[1]{\color{gray}\begin{enumerate}[nosep,leftmargin=*]\item[] #1\end{enumerate}\color{black}} % Respuestas
\newcommand{\encabezado}[3]{\large
Universidad de Costa Rica						\hfill Escuela de Ciencias de la Computación e Informática  \\
Sistemas Operativos (CI-1310)	\hfill \textbf{Tarea \##1}                 \\
\textbf{Profesor:} Francisco Arroyo Mora		\hfill Grupo 1                                              \\
\textbf{Estudiante:} Daniel Marín Montero B44007\hfill #2/#3/2018                                           \\
\rule{196mm}{1pt}\normalsize

\tikz[remember picture,overlay] \node[opacity=0.3,inner sep=0pt] at (0.93cm,1.43cm){\includegraphics[width=1.82cm,height=1.9cm]{pictures/UCR.png}};
\tikz[remember picture,overlay] \node[opacity=0.3,inner sep=0pt] at (18.65cm,1.43cm){\includegraphics[width=1.69cm,height=1.9cm]{pictures/ECCI.png}};

\vspace{-0.4cm}} % Encabezado de los documentos


\begin{document}
\encabezado{1}{5}{10}
\section{Introducción}
El presente documento tiene como objetivo informar al usuario como utilizar el programa requerido y mostrar su funcionamiento por medio de las pruebas utilizadas. El trabajo principal del programa es crear n procesos que tomen un archivo con extensión XML cada uno y aislar sus etiquetas del contenido contando cuantas veces se repiten, enviarle mensajes al proceso principal con los datos obtenidos para que este los comparta en un segmento de memoria compartida y sean impresos por otro proceso en pantalla. Este programa por medio de un parámetro -t define si se debe de crea un archivo aparte que contenga todos los datos entre las etiquetas. 

\section{Manual de Usuario}
\subsection{Requerimientos de Software}
	\begin{itemize}
    \item	\textbf{Sistema Operativo:} [ Linux ]
    \item	\textbf{Arquitectura:} [ 32 bits, 64 bits ]
    \item	\textbf{Ambiente:} [ gedit, nano ]
    \end{itemize}
\subsection{Compilación}
\begin{itemize}
    \item[]Para compilar el programa hay que ubicarse donde se encuentra el archivo makefile dentro de la terminal, para ejecutar el siguiente comando:
\begin{lstlisting}[style= Cstyle]
	make compile
\end{lstlisting}
  	Lo anterior llama al compilador g++ de la siguiente manera: 
\begin{lstlisting}[style= Cstyle]
	g++ -g -std=c++11 src/*.cpp -o bin/labelCounter.out
\end{lstlisting}
    \end{itemize}
\subsection{Especificación de las funciones del programa}
\begin{itemize}   
	\item[] Este programa requiere de archivos con extensión XML, en caso de no brindar ninguna ruta válida el programa lo informará y procederá a cerrarse.
    
	Para correr el programa se puede utilizar el siguiente comando desde la ruta del make:
\begin{lstlisting}[style= Cstyle]
	./bin/labelCounter.out [-t] [-s] <ruta1> <ruta2> ... <rutaN>
\end{lstlisting} 
	Donde:
 -t indica que se requiere crear un archivo de salida por cada archivo de entrada con el contenido entre las etiquetas del archivo XML, el archivo creado se guardará en la misma ubicación que el archivo de entrada con el mismo nombre pero sin extensión.

-s indica que se quieren sobreescribir todos los archivos de salida, en caso de que existan, 

	Alternativamente se puede utilizar el commando:
\begin{lstlisting}[style= Cstyle]
	make run
\end{lstlisting}
    Lo que llamará al comando:
\begin{lstlisting}[style= Cstyle]
	./bin/labelCounter.out -t -s data/input1.xml data/input2.xml
\end{lstlisting}
\end{itemize}
\section{Pruebas}

\section{Conclusión}
    
\end{document}
